\documentclass[twoside,spanish]{elsarticle}
\usepackage[T1]{fontenc}
\usepackage[latin9]{inputenc}
\pagestyle{headings}
\usepackage{float}
\usepackage{amstext}
\usepackage{amsthm}
\usepackage{amssymb}
\usepackage{graphicx}
\PassOptionsToPackage{normalem}{ulem}
\usepackage{ulem}

\makeatletter

\floatstyle{ruled}
\newfloat{algorithm}{tbp}{loa}
\providecommand{\algorithmname}{Algoritmo}
\floatname{algorithm}{\protect\algorithmname}

\theoremstyle{plain}
\newtheorem{thm}{\protect\theoremname}
\theoremstyle{definition}
\newtheorem{defn}[thm]{\protect\definitionname}

\usepackage{algorithm}
\usepackage{algpseudocode}

\journal{Curso de <<An�lisis de algoritmos>>, PUJ, Bogot�, Colombia - }


\makeatother

\usepackage{babel}
\addto\shorthandsspanish{\spanishdeactivate{~<>}}

\providecommand{\definitionname}{Definici�n}
\providecommand{\theoremname}{Teorema}

\begin{document}

\begin{frontmatter}{}

\title{Algoritmo para solucionar las ``Torres de Hanoi''\tnoteref{t1}}

\tnotetext[t1]{Este documento presenta la escritura formal de un algoritmo que soluciona
el juego de las Torres de Hanoi.}

\author[lfv]{Leonardo Fl�rez-Valencia}

\ead{florez-l@javeriana.edu.co}

\address[lfv]{Pontificia Universidad Javeriana, Bogot�, Colombia}

\begin{abstract}
En este documento se presenta la escritura formal del problema ``calcular
la secuencia de pasos para resolver el juego de las Torres de Hanoi'',
junto con un algoritmo de soluci�n.
\end{abstract}

\begin{keyword}
algoritmo, escritura formal, Torres de Hanoi.
\end{keyword}

\end{frontmatter}{}

\section{An�lisis del problema}

Se desea escribir un algoritmo informar la secuencia de pasos que
resuelven el juego de las torres de Hanoi. Antes de proponer dicha
soluci�n algor�tmica, primero es importante describir el juego.

\begin{defn}
Las Torres de Hanoi es un rompecabezas o juego matem�tico inventado
en 1883 por el matem�tico franc�s �douard Lucas. Es un juego de un
jugador que consiste en un n�mero de discos perforados de radio creciente
que se apilan insert�ndose en uno de las tres torres fijadas a un
tablero. El objetivo del juego es trasladar la pila a otro de los
postes siguiendo las reglas:
\end{defn}

\begin{enumerate}
\item solo se puede mover un disco a la vez y
\item no se puede colocar un disco m�s grande encima de un disco m�s peque�o
\end{enumerate}

Un esquema de un estado del juego se presenta a continuaci�n:
\begin{center}
\includegraphics[width=0.75\textwidth]{hanoi_schema}
\par\end{center}

De esta descripci�n del juego se puede concluir:
\begin{enumerate}
\item La cantidad de discos es finita y contable, entonces es un n�mero
natural $n\in\mathbb{N}$.
\item El radio exacto de cada disco es irrelevante, lo �nico relevante es
un dato que permita ordenar los discos, entonces el radio de cada
disco se puede representar con un n�mero natural $r\in\mathbb{N}$.
\end{enumerate}

\section{Dise�o del problema}

El an�lisis anterior nos permite dise�ar el problema: definir las
entradas y salidas de un posible algoritmo de soluci�n, que a�n no
est� definido.
\begin{enumerate}
\item \emph{\uline{Entradas}}:
\begin{enumerate}
\item $n\in\mathbb{N}$, la cantidad de discos.
\item $\left(o,d,a\right)$, una tripleta con los identificadores de las
torres: \textbf{\emph{\uline{o}}}rigen, \textbf{\emph{\uline{d}}}estino
y \textbf{\emph{\uline{a}}}uxiliar.
\end{enumerate}
\item \emph{\uline{Salidas}}: $S=\left\langle \left(i,j\right);i,j\in\left\{ o,d,a\right\} \right\rangle $
la secuencia de pasos para solucionar el juego.
\end{enumerate}

\section{Algoritmo de soluci�n}

El algoritmo de soluci�n se basa en una estrategia de soluci�n recurrente,
donde se busca mover todos los discos superiores a la torre auxiliar,
para luego mover el disco m�s grande al destino.

\begin{algorithm}
\begin{algorithm}[H]

\begin{algorithmic}[1]

\Require{$n\in\mathbb{N}$, the number of disks to move}

\Require{$t\equiv\left(o,d,a\right)$, a tuple with the three towers
(\textbf{\emph{\uline{o}}}rigin, \textbf{\emph{\uline{d}}}estination
and \textbf{\emph{\uline{a}}}uxiliary)}

\Procedure{SolveTowersOfHanoi}{$n,t\equiv\left(o,d,a\right)$}

  \If{$n=1$}

    \State\Return$\left\langle t\right\rangle $

  \Else

    \State$M\leftarrow\emptyset$

    \State$M\leftarrow M\bigcup\text{\Call{SolveTowersOfHanoi}{\ensuremath{n-1,\left(o,a,d\right)}}}$

    \State$M\leftarrow M\bigcup\text{\Call{SolveTowersOfHanoi}{\ensuremath{1,\left(o,d,a\right)}}}$

    \State$M\leftarrow M\bigcup\text{\Call{SolveTowersOfHanoi}{\ensuremath{n-1,\left(a,d,o\right)}}}$

    \State\Return$M$

  \EndIf

\EndProcedure

\end{algorithmic}

\end{algorithm}

\caption{Solucionador de las Torres de Hanoi}

\end{algorithm}


\subsection{Invariante}

La secuencia de movimientos se actualiza respetando las reglas del
juego.

\subsection{An�lisis de complejidad}

El algoritmo exhibe una complejidad $O\left(2^{n}\right)$.

\subsection{Notas de implementaci�n}

La secuencia de retorno $M$ debe ser implementada con una estructura
lineal estilo ``lista de tuplas''. Los identificadores de las torres
deben ser representados por valores finitos contables (n�meros naturales
o enteros o cadenas de caracteres).
\end{document}
